\documentclass[english, 11pt]{article}
% \usepackage[T1]{fontenc}
% \usepackage[latin9]{inputenc}
\usepackage[top = 2cm, left = 2.5cm, right = 2.5cm, bottom = 2cm]{geometry}
\usepackage{enumitem}
\usepackage{amsmath}
\usepackage{amsfonts}
\usepackage{amstext}
\usepackage{mdframed}
\usepackage{graphicx}
% \usepackage{bbm}
\makeatletter
\@ifundefined{date}{}{\date{}}
\usepackage{tikz}
\usetikzlibrary{quotes, angles, decorations.markings, intersections}
\usetikzlibrary{calc,patterns,angles,quotes, 3d, intersections, positioning, shapes, automata, positioning}
\usepackage{wasysym}
\makeatother
\usepackage{babel}
\usepackage{color}
\usepackage{graphicx}
\usepackage{hyperref}
\hypersetup{
	colorlinks,
	citecolor=black,
	filecolor=black,
	linkcolor=black,
	urlcolor=black
}


\newcommand{\tbox}[1]{\noindent\fbox{\parbox{\textwidth}{#1}}}

\setlength{\parindent}{0pt}

\begin{document}

\textbf{CS 745 : Principles of Data and System Security} \hfill DATE
\textit{Scribed by}: Kavin Arvind

\noindent\tbox{

\begin{center}
  \huge Lecture - 01 \\ % change lecture number
  \Large Topic:  History of Cryptography
\end{center}

}

\section{Shifted cipher}
Each letter is Shifted by $k$ and sent. Eg- "A" is written as "A"+k (Shifted by k letters) and sent. \\
This is easy to decode as only 26 ( or 36 (if 0-9 nos are included)) possible $k$ are there and thus its easy to check each possibility.

\section{Rolling by wooden stick}
A paper is rolled on to a stick and text is written. If seen normally, the letters would look fully shuffled, but if its rolled in the same way as it was written, it can be decoded.\\
eg- "MY NAME IS X" is written like
\begin{figure}[ht]
    \centering
        \begin{tabular}{ccccccccc}
            M &   &   &   & M &   &   &   & X\\
              & Y &   & A &   & E &   & S &  \\
              &   & N &   &   &   & I &   &  \\
        \end{tabular}
\end{figure}
\\
Thus its crypted as MMXYAESNI.

\section{Substitution cipher}
We have a table where each letter is mapped to other letters and text is ciphered according to that. Here, we have here $26!$ ways of mapping and so its very difficult to try different possibilities. \\
This seems like an optimal solution, but there is a problem. In an average english text, each letter has a specific frequency of repetition. \\
Say letter "A" is coded to letter "K" (randomly). So frequency of letter K would be same as of the letter "A" in a normal text. So by this way, cipher text could possibly be decrypted.

\end{document}